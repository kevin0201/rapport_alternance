\addcontentsline{toc}{chapter}{Conclusion}
\chapter*{Conclusion}

Python a été un véritable compagnon au cours de cette expérience professionnel. L'ensemble des missions ont été menées avec lui comme principal outil. Il se veut être une véritable alternanive, souple, puissant et polyvalant, à Microsoft Excel et SAS.
\\
\\
L'ensemble des missions sur lesquelles j'ai travaillé m'ont appris le sens de l'organisation, de l'objectivité et de l'autonomie. Il est nécessaire de connaître son objectif et de veiller à ne pas s'en éloigner. Certaines missions sont plus difficiles que d'autres, et certaines peuvent identifier des faiblesses existantes. Le monde professionnel est fait de challenge continuel, il faut pas hésiter remettre ses méthodes en question de même que ses approches. Certaines missions peuvent mener à des impasses, toutefois contournable, mais avec un coût de chantier en temps et en travail, plus élevé que prévu. C'est aussi le monde professionnel, dans un sens mathématique on dira que tout n'est pas linéaire et continue.
\\
\\
La méthode de détection d'anomalie par isolation est une méthode qui comme énoncée, se veut nouvelle et innovante, elle se démarque des méthodes basée sur les distances par une approche essentiellement stochastique. De son application, on obtient des résultats plutôt réalistes et convaincant. En plus elle fait économiser sur les coûts de calcul de distance et de stockage de matrice de distance énormes. Cependant, penser qu'elle est parfaite serait utopique. Il est important de l'appliquer sur les bonnes variables pour avoir les bonnes réponses tout en tenant compte du fait qu'elle puisse passer à côté de certaines anomalies.