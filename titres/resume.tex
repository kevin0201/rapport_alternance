\addcontentsline{toc}{chapter}{Résumé}
\chapter*{Résumé} 

\textbf{Mots clés~:} \emph{python, cython, reporting, participation, arbre, isolation, scores, anomalie, forme, distribution}.
\\
\\
Les équipes techniques au sein de sociétés d'assurances utilisent tout un ensemble d'outils et de méthodes pour accomplir des tâches simples et complexes. Dans un contexte d'évolution dynamique, ces outils et méthodes ont tout aussi besoin d'évoluer. C'est dans ce contexte que python et l'apprentissage automatique arrivent en assurance vie, pour apporter de la performance de l'efficacité et de l'innovation. 
\\
\\
Ce rapport a pour objectif de présenter l'ensemble des missions transverses qui ont pu être exécutées avec python. Il montre comment python se veut un outil, utile et puissant. Python n'est plus un simple langage de programmation, il devient un sérieux concurrent de SAS en terme de manipulation et de d'analyse de données. En outre, il a aussi permit de construire un modèle de détection d'anomalies, basé sur le concept de machine learning, pour la fiabilisation de données. Python dispose d'une multitude de bibliothèques qui font de lui, un langage complet.

